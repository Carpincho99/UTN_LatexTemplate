\usepackage{color}
\usepackage{listings} %CODE STYLE 

\lstloadlanguages{C,C++,csh,Java}

\definecolor{cred}{rgb}{0.6,0,0} 
\definecolor{cblue}{rgb}{0,0,0.6}
\definecolor{cgreen}{rgb}{0,0.5,0}
\definecolor{ccyan}{rgb}{0.0,0.6,0.6}
\definecolor{ccloudwhite}{rgb}{0.9412, 0.9608, 0.8471} 
\definecolor{cdavysgrey}{rgb}{0.33, 0.33, 0.33}
\definecolor{cdeepfuchsia}{rgb}{0.76, 0.33, 0.76}
\definecolor{cdeepskyblue}{rgb}{0.0, 0.75, 1.0}
\definecolor{cdeepsaffron}{rgb}{1.0, 0.6, 0.2}
\definecolor{cdeeppink}{rgb}{1.0, 0.08, 0.58}
\definecolor{cdodgerblue}{rgb}{0.12, 0.56, 1.0}

\lstset{literate=%colorizar numeros
   *{0}{{{\color{cdeepfuchsia}0}}}1
    {1}{{{\color{cdeepfuchsia}1}}}1
    {2}{{{\color{cdeepfuchsia}2}}}1
    {3}{{{\color{cdeepfuchsia}3}}}1
    {4}{{{\color{cdeepfuchsia}4}}}1
    {5}{{{\color{cdeepfuchsia}5}}}1
    {6}{{{\color{cdeepfuchsia}6}}}1
    {7}{{{\color{cdeepfuchsia}7}}}1
    {8}{{{\color{cdeepfuchsia}8}}}1
    {9}{{{\color{cdeepfuchsia}9}}}1
}

\lstdefinelanguage{pseudo}{
    morekeywords = [1]{INICIO, FIN},
    morekeywords = [2]{Imp, Leer},%i/o
    morekeywords = [3]{Si, No, Sino, Para, Cuando, Mientras, Hacer, Que}, 
    keywordstyle = [1]\color{cdeeppink},
    keywordstyle = [2]\color{cdodgerblue},
    keywordstyle = [3]\color{cblue},
    sensitive = true,
    morecomment = [l]{//},
    morecomment = [s]{/*}{*/},
    morecomment = [s]{/**}{*/},
    commentstyle=\color{cgreen},
    morestring = [b]",
    morestring = [b]',
    stringstyle=\color{cred}\ttfamily,
}

\lstset{
    language=csh,
    basicstyle=\footnotesize\ttfamily,
    numbers=left,
    numbersep=5pt,
    tabsize=2,
    extendedchars=true,
    breaklines=true,
    frame=single,
    showspaces=false,
    showtabs=false,
    xleftmargin=17pt,
    framexleftmargin=17pt,
    framexrightmargin=5pt,
    framexbottommargin=4pt,
    morecomment=[l]{//}, %use comment-line-style!
    morecomment=[s]{/*}{*/}, %for multiline comments
    showstringspaces=false,
    morekeywords={ abstract, event, new, struct,
        as, explicit, null, switch,
        base, extern, object, this,
        bool, false, operator, throw,
        break, finally, out, true,
        byte, fixed, override, try,
        case, float, params, typeof,
        catch, for, private, uint,
        char, foreach, protected, ulong,
        checked, goto, public, unchecked,
        class, if, readonly, unsafe,
        const, implicit, ref, ushort,
        continue, in, return, using,
        decimal, int, sbyte, virtual,
        default, interface, sealed, volatile,
        delegate, internal, short, void,
        do, is, sizeof, while,
        double, lock, stackalloc,
        else, long, static,
        enum, namespace, string},
    commentstyle=\color{cgreen},
    keywordstyle=\color{cblue},
    stringstyle=\color{cred}\ttfamily,
    numberstyle=\tiny\color{cdeepfuchsia},
    identifierstyle=\color{black},
    backgroundcolor=\color{white},
}

%Nombre: codef
%Funcion: Insertar codigo desde un archivo.
%Parametros:
% 1. Ruta del archivo.
% 2. Lenguaje ej: "c"
% 3. Etiqueta.
% 4. Descripcion.
\newcommand{\code}[4]{
    \lstinputlisting[language=#2, label={#3}, caption={#4}, captionpos=b]{#1}
}

\renewcommand{\lstlistingname}{Código}% Listing -> Código